\section{Условие}
Родительский процесс создает два дочерних процесса. Первой строкой пользователь в консоль \
родительского процесса вводит имя файлов, которые будут использованы для открытия файлов с \
таким именем на запись. Родительский и дочерний процесс должны быть представлены разными про- \
граммами. Родительский процесс принимает от пользователя строки произвольной длины и \
пересылает их в дочерние процессы через системные сигналы/события и/или через отображаемые \
файлы (memory-mapped files). Дочерние процессы удаляют из строк гласные буквы и записывают \
строки в файлы, заданные в начале. \
Родительский процесс полученные от child \
ошибки выводит в стандартный поток вывода.

\subsection*{Цель работы}
Изучение механизмов создания процессов, организации межпроцессного взаимодействия через отображаемые \
файлы (memory-mapped files) и обработки данных в многопроцессной архитектуре.

\subsection*{Задание}
Дочерние процессы удаляют из строк гласные буквы и записывают \
строки в файлы, заданные в начале.

\subsection*{Вариант} 18